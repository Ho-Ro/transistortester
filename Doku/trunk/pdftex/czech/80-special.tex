\chapter{Speciální součásti softwaru}
Bylo provedeno mnoho změn ukládání do paměti flash.
Výstup LCD testovacích čísel portů byl proveden formou ,,lcd\_data('1'+Pin)''.
\\K ušetření místa pro další požadavky při každé výzvě, byla funkce ,,lcd\_testpin(uint8\_t pin)'' vložena do souboru lcd\_routines.c.

Pseudo funkce v podobě \_delay\_ms(200) nejsou funkce knihovny,
místo toho jsou v programu pro každý bod spojení vestavěny čekací smyčky.
To spotřebovává spoustu paměti, pokud je mnoho výzev do různých míst programu.

Všechny tyto výzvy byly vyměněny za speciální výzvu v assembleru napsanou knihovnou,
která spotřebuje pouze 74 ~ bajtů paměti flash (u \(8MHz\)),  ale
dává k disposici výzvy od wait1us() do wait5s() v úrovních 1,2,3,4,5,10,20\dots.

Pro všechny výzvy přes \(50ms\) obsahují rutiny příkaz ''Watch Dog Reset''.
\\Každá výzva vyžaduje pouze jednu instrukci (2~Byte). Čekající výzvy s přechodnými hodnotami
jako 8 ms potřebují dvě výzvy (wait5ms() a wait3ms() nebo dva wait4ms() výzvy).

Neznám žádné řešení, které by bylo ekonomičtější, pokud je v programu použito mnoho výzev.
Výzvy nepoužívají registry, pro návratové adresy do RAM, pouze ukazatel zásobníku (stack pointer)
v paměti RAM (maximálně 28 bajtů).

Úplný seznam funkcí je:\\
wait1us(), wait2us(), wait3us(), wait4us(), wait5us(), wait10us(), \\
wait20us(), wait30us(), wait30us(), wait40us(), wait50us(), wait100us(), \\
wait200us(), wait300us(), wait400us(), wait500us(), wait1ms(),\\
wait2ms(), wait3ms(), wait4ms(), wait5ms(), wait10ms(),\\
wait20ms(), wait30ms(), wait40ms(), wait50ms(), wait100ms(),\\
wait200ms(),wait300ms(), wait400ms, wait500ms(), wait1s(),\\
wait2s(), wait3s(), wait4s() und wait5s();\\

Jedná se o 36 funkcí, které používají pouze 37 instrukcí včetně Watch Dog Reset!
\\Neexistuje žádný způsob, jak tuto knihovnu zkrátit.\\

V neposlední řadě dodržují tyto výzvy přesný čas, když to dělá nejnižší výzva (wait1us).
\\Pouze čekající výzvy přes \(50ms\) jsou jeden takt pro \(100ms\) delší, kvůli dodatečně zabudovanému
Watch Dog resetu.


Kromě toho byla často používá sekvence ,,wait5ms(); ReadADC(\dots);'' nahrazena jedinou výzvou ,,W5msReadADC(\dots);''.

Totéž platí pro sekvenci ,,wait20ms(); ReadADC(\dots);'' provedené výzvou
,,W20msReadADC(\dots);''.
Funkce ReadADC byla dodatečně přenesena do assembleru, takže toto rozšíření
může být velmi účinně integrováno.

K dispozici je také funkčně identická C-verze ReadADC funkce.
