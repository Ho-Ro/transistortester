\chapter{Pracovní úkoly a nové nápady}
\label{sec:todo}
\begin{enumerate}
\item Přidat další a lepší dokumentaci.
\item Přemýšlet o tom, jak určit skutečný vnitřní odpor vývodů portu B- namísto předpokladu, že jsou porty  stejné.
\item Může být urychleno vybíjení kondenzátorů, kdyby byl kromě toho záporný pól přes \(680\Omega\) odpor
spojen s VCC (+)?
\item Zkontrolovat, zda tester může používat hodnoty s plovoucí desetinnou čárkou.
Riziko přetečení (overflow) je nižší.
Nebylo by třeba konstrukce násobení nebo dělení k modelu faktoru s částečným číslem.
Ale nevím, kolik místa potřebuje knihovna.
\item Napsat návod k instalaci testeru s volbami makefile a popsat průběh procesu až k dokončenému procesoru.
\item Pokud  nemůže být dosažen držící proud tyristoru pomocí \(680\Omega\) odporu. Je pro velmi krátkou dobu bezpečné přepnout katodu přímo na GND a anodu přímo do VCC?
\\Proud může dosáhnout více než \(100mA\). Bude port poškozen? A co napájení (regulátor napětí)?
\item Po této akci zkontrolovat porty s funkcí autotestu!
\item Nápad na nový projekt: USB verze bez LCD displeje, napájení z USB portu, komunikace s PC přes USB-Serial  most.
\item Výměna funkce samplingADC s využitím Counter1 k upravení časového posunu ADC S\&H.
\item Zkouška přesnosti měření pro malé kondenzátory metodou SamplingADC.
\item Vyšetřování přesnosti pro malé cívky pomocí metody SamplingADC.
\end{enumerate}
