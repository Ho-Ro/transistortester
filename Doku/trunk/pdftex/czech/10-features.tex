%\newpage
\chapter{Vlastnosti}
\label{sec:features}
\begin{enumerate}
\item Pracuje s mikrokontroléry ATmega8, ATmega168 nebo ATmega328.Ale lze použít také ATmega644, ATmega1280, ATmega1284 nebo ATmega2560 .
\item K zobrazení výsledků měření je vhodný LCD displej 2x16 nebo 4x20 znaků.
 Alternativně může být, při použití procesoru s alespoň 32K flash pamětí, také grafický displej
 s 128x64 pixely a řadiče ST7565, ST7920, NT7108, KS0108 nebo SSD1306.
 Místo standardního 4bitového paralelního rozhraní je možné použít buď 4-vodičové rozhraní SPI ale také I\textsuperscript{2}C sběrnici.
 Dokonce lze použít také barevné displeje s ILI9163 nebo ST7735 řadiči s SPI rozhraním.
 Ovladač NT7108 nebo KS0108 vyžaduje sériově paralelní převodník 74HC (T) 164 nebo 74HC (T) 595,
 protože tyto obvody umožňují pouze 8bitové paralelní připojení,
 Displeje s ovladači PCF8812 nebo PCF8814 lze použít pouze bez velkých tranzistorových symbolů
 proto že je velikost jejich zobrazení nedostatečná (102x65 a 96x65).
\item Jednoduchý provoz s funkcí automatického vypnutí.
\item Provoz s akumulátorem je možný, protože spotřeba po vypnutí je jen asi 20nA.
Od softwarové verze 1.05k používá ATmega k úsporám energie během přestávek měření stav spánku, pokud právě není používán rotační enkodér.
\item Je možné i levné řešení bez krystalu a bez automatického vypínání.
\item Automatická detekce NPN a PNP bipolárních tranzistorů, N- a P-KANÁLOVÝ MOSFET, JFETs, diody, dvojité diody, N- a P-IGBT, tyristory a triaky.
Pro tyristory a triaky musí být dosaženo dostatačné zapalovací a udržovací napětí a proudy.
U IGBT musí být prahové napětí brány nižší než  \(5V\) .
\item Znázornění rozložení pinů testovacích součástek.
\item Měření stávajícího zesilovacího činitele a prahového napětí báse-emitor pro bipolární tranzistory.
\item Darlingtonovy tranzistory jsou charakteristické vyšším prahovým napětím a vysokým proudovým zesílením.
\item Automatická detekce ochranné diody v bipolárních tranzistorech a MOSFETů.
\item Měření prahového napětí, vstupní kapacity a R\textsubscript{DSon} s hradlovým napětím těsně pod \(5V\) u MOSFETů.
\item Jsou měřeny a zobrazeny až dva odpory jako \mbox{~\electricR} symboly a jejích hodnoty jsou až na čtyři desetinná místa ve správné hodnotě.
Všechny symboly jsou zarámovány s testovacími čísly, jak byly nasazeny do zkoušečky (1-3).
Proto lze také měřit potenciometry. Když ale potenciometr dosáhne koncové polohy,
Není možné rozlišit mezi prostředním a koncovým kontaktem.
\item Odpory lze nyní měřit od \(0,01\Omega\), do \(50M\Omega\) .
\item Kondenzátor je také detekován a změřen. Je označen symbolem \mbox{\electricC}
Jeho kapacita je určena a zobrazena až na čtyři desetinná místa přesně.
Hodnota může být v rozmezí od \(25pF\) (při \(8MHz\) taktu, \(50pF\) při \(1MHz\) taktu) do \(100mF\) . Rozlišení je \(1pF\) (u \(8MHz\) taktu) .
\item U kondenzátorů s kapacitou větší než \(20nF\) je kromě toho měřen ještě ekvivalentní sériový odpor (ESR) kondenzátoru
s rozlišením \(0,01\Omega\) a zobrazen na dvě desetinná místa.
Tato funkce je k dispozici pouze tehdy, pokud má ATmega nejméně 16K flash paměťi.
\item U kondenzátorů s kapacitní hodnotou nad \(5000pF\) lze po nabíjecím impulsu určit ztrátovou hodnotu Vloss.
Ztrátová hodnota v procentech indikuje kvalitu kondenzátoru.
\item Až dvě diody jsou označeny symbolem \mbox{\electricDAK} nebo symbolem \mbox{\electricDKA}
a jsou zobrazeny ve správném pořadí .
Kromě toho jsou zobrazeny úbytky napětí na diodách.
\item LED dioda je rozpoznána jako dioda, úbytek napětí je ale mnohem vyšší než u normální diody.
Dvojité diody jsou rozpoznány jako dvě diody.
\item Zenerovy diody lze detekovat, když je Zenerovo napětí pod hodnotou \(4,5V\) .
Zobrazují se jako dvě diody, rozpoznat je lze jen přes zobrazené napětí.
Vnější čísla zkušebního kontaktu obklopující symboly diod jsou v tomto případě totožné.
Skutečnou anodu diody lze nalézt pouze pro diodu, jejíž prahové napětí je blízké napětí \(700mV\) !
\item Pokud se zjistí více než 3 diody, zobrazí se spolu s chybovou zprávou počet nalezených diod.
K tomu může dojít pouze v případě diod na všech třech zkušebních pinech a jsou spojeny a alespoň jedna z nich je Zenerova dioda. V tomto případě je třeba připojit pouze dva testovací kontakty a restartovat skenování a měřit jednu diodu za druhou.
\item Kapacita diody v závěrném směru je určena automaticky.
Bipolární tranzistory lze také testovat, pokud je připojena pouze báze a buď kolektor nebo emitor.
Pro ATmega s více než 8k flash pamětí je kromě toho měřen ještě zpětný proud s rozlišením \(2nA\) .
Hodnota je zobrazena pouze tehdy pokud je rozdílná od nuly.
\item Zapojení usměrňovacího můstku lze zjistit pouze jedním měřením.
\item Kondenzátory s hodnotami kapacity pod \(25pF\) není možné běžně rozpoznat, 
ale mohou být použity společně s diodou zapojenou paralelně nebo s paralelně připojeným kondenzátorem
kapacity nejméně \(25pF\) .
V tomto případě musí být od výsledku měření odečtena hodnota kapacity součásti zapojené paralelně.
U procesorů s minimální pamětí 32K flash se tester změní pomocí kondenzátoru \textgreater~\(25pF\)
mezi TP1 a TP3 v cyklické měření kondenzátoru, která také přímo měří kapacity od \(1pF\) .
\item Pro odpory pod \(2100\Omega\) se také provádí měření indukčnosti pokud
má ATmega nejméně 16K flash paměťi.
Kromě symbolu odporu \mbox{~\electricR} se zobrazí symbol indukčnosti \mbox{~\electricL} .
Rozsah zobrazení je asi \(0,01mH\) až přes \(20H\), ale přesnost není vysoká.
Výsledek se zobrazuje pouze pro jeden rezistor společně s hodnotou odporu.
\item Doba měření je asi dvě sekundy, měření kapacity a indukčnosti mohou trvat déle.
\item Software lze konfigurovat pro sérii měření s předem definovaným počtem opakování, než se automatické vypne.
\item Vestavěná funkce automatického testování včetně volitelného frekvenčního generátoru \(50Hz\)  pro přesnost kontroly frekvence
a časové prodlevy (pouze s minimálně 16 kB flash pamětí).
\item Volitelná možnost kalibrace pro měření kondenzátoru a vnitřní odpor pro
automatické určování portů během samočinného testu (pouze s minimálně 16 kB flash pamětí).
Externí kondenzátor s kapacitou mezi \(100nF\) a \(20\mu F\) na testovacích kontaktech TP1 a TP3 je nutný, 
pro kompenzaci vyrovnávacího napětí analogového komparátoru.
To může snížit chybu měření při měření kapacity až na hodnotu \(40\mu F\) .
Stejným kondenzátorem je korekční napětí pro nastavení správného zesílení pro
vypočet měření ADC pomocí vnitřního referenčního napětí \(1,1V\) .
\item Zobrazení kolektor - emitor zbytkového proudu \(I_{CE0}\) při odpojené bázi (\(1\mu A\) přesnost) 
a zbytkový proud kolektor - emitor  \(I_{CES}\) s bází připojenou na potenciál emitoru (pouze s minimálně 16K flash paměťí).
Tyto hodnoty se zobrazují pouze v případě, že nejsou nulové (zejména pro germaniové tranzistory).
\item Pro ATmega s minimálně 32K flash pamětí se tester přepne z multifunkčního testu na režim
měřiče odporu, pokud je v automatickém režimu rozpoznání součástek, pouze jeden odpor na testovacích kontaktech (TP1) a (TP3). Pokud je v souboru Makefile zapnuto pomocí volby RMETER\_WITH\_L při měření odporů také měření indukčnosti, měří se také. Provozní režim je indikován s {\bf[R]} nebo {\bf[RL]} na pravé straně 1 řádku displeje. Přesně tak, jak se tester přepne na měřič kapacit, když byl mezi TP1 a TP3 detekován kondenzátor.
Tento provozní režim je označen symbolem {\bf[C]} na pravé straně 1 řádku displeje.
V tomto režimu lze měřit kondenzátory  od \(1pF\) . Pouze pro automatické spuštění funkce
potřebujete kondenzátor s více než \(25pF\).
Obě speciální funkce lze opět ukončit stisknutím tlačítka. Tester poté funguje v normálním režimu.
\item U procesorů s min. 32 kB flash pamětí je přístupné po dvousekundovém stisku tlačítka menu, což zprovozní další funkce. Menu lze samozřejmě použít také k návratu k funkci testeru tranzistoru.
\item Pomocí funkce menu lze provést měření frekvence na portu PD4 ATmega.
Rozlišení je  \(1Hz\) na vstupních frekvencích nad \(33kHz\) .
Při nižších frekvencích může být rozlišovací schopnost až \(0,001mHz\) .
Přečtěte si prosím podkapitolu \ref{sec:frequency_counter} na stránce \pageref{sec:frequency_counter},
jak musí být frekvenční signál připojen.
\item Pomocí funkce menu a při vypnutí UART módu lze měřit externí napětí do 50 V přes 10:1 dělič napětí na PC3 kontaktu. U PLCC-ATmega328 varianty je možné jeden z těch dvou přidaných kontaktů dohromady s UART rozhraním použít na měření napětí. Pokud je připojené rozšíření pro měření Zenerovy diody (převodník DC-DC), je možné v této větvi, při současném podržení tlačítka, testovat Zenerovy diody.
\item Pomocí další funkce menu lze na kontaktu TP2 (PB2 port ATmega) zapnout výstup frekvence.
V současné době lze frekvence nastavit od \(1Hz\) do \(2MHz\) .
\item Pomocí další funkce menu lze zapnout na pinu TP2 (PB2 port ATmega) pevně danou frekvenci s nastavitelnou šířkou impulsu.
Šířku impulzu lze zvýšit o \(1\%\) krátkým stiskem klávesy a o \(10\%\) s delším stiskem.
\item Pomocí další funkce menu lze spustit speciální měření kondenzátoru s měřením ESR.
Tato funkce se při výběru nazývá \mbox{C+ESR@TP1:TP3} .
 Kapacity od přibližně \(2\mu F\) až do \(50mF\) mohou být měřeny pro nízké měřicí napětí okolo \(300mV\)
 v zapájeném stavu.
\item Pro procesory s alespoň 32K flash pamětí (Mega328) lze ADC použít s metodou vzorkování,
která umožní měřit kondenzátory pod \(100pF\) s rorlíšením  \(0,01pF\) .
Stejným způsobem je možné měřit také cívky pod \(2mH\) čímž lze dosáhnout výrazně lepší rozlišení
než rezonanční frekvence s paralelním kondenzátorem známé velikosti.
\end{enumerate}

 Při použití testeru pro testování kondenzátorů v obvodu je třeba věnovat zvláštní pozornost tomu
  aby kondenzátory neměly, již před měřením, žádné zbytkové napětí.

Tyristory a triaky lze detekovat pouze tehdy, je-li testovací proud nad přídržným proudem.
Některé tyristory a triaky také vyžadují vyšší zapalovací proud, než tento tester dokáže dodat.
Dostupný testovací proud je pouze asi \(6mA\)!
Stejně tak mohou být IGBT detekovány pouze tehdy, je-li zkušební napětí asi \(5V\) pro testování dostatečné.
Vezměte prosím na vědomí, že víc možností je k dispozici pouze u mikroprocesorů s minimálně 16 kB programovou pamětí, jako je ATmega168. 
Všechny funkce jsou dokonce možné pouze u procesorů s programovou pamětí alespoň 32K, jako ATmega328 nebo ATmega1284.

\vspace{1cm}
\textbf{{\Large Pozor:}} Vždy zkontrolujte, zda jsou {\bf Kondenzátory}  před připojením ke zkoušečce,
nejlépe zkratováním, {\bf vybité} !
V opačném případě by mohlo dojít k poškození přístroje ještě před jeho zapnutím.
! ATmega nabízí jen málo vlastní ochrany. !
Zvláštní pozornost je třeba věnovat také při měření v zapojení.
Přístroj by měl být vždy předem odpojen od napájení a měli byste se přesvědčit, že v přístroji není {\bf žádné zbytkové napětí} .
