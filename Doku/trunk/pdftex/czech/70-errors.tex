\chapter{Známé chyby a nevyřešené problémy}
\vspace*{-1cm}
{\center Software-Version 1.12k}

\begin{enumerate} \setlength{\itemsep}{0em}

\item Germanium diody (AC128) nejsou ve všech případech detekovány. Příčina je pravděpodobně zbytkový proud.
Chlazení diody snižuje zbytkový proud.

\item V bipolárních tranzistorech není detekována ochranná dioda kolektoru - emitoru
Zbytkový proud kolektoru ICE0 je vysoký. Doposud byl problém zjištěn pouze s germania tranzistory s
externí diodou.

\item Aktuální faktor zesílení germaniových tranzistorů může být měřen příliš vysokým z důvodu vysokého zbytkového proudu.
V tomto případě je naměřené napětí báze emitor nápadně malé.
Chlazení tranzistoru může pomoci určit realističtější zesílení proudu.

\item U dvojitých diod typu Schottky, jako je MBR3045PT, nelze detekovat kapacitu v opačném směru. Důvodem je příliš vysoký zbytkový proud. Chybě může být někdy zabráněno chlazením (studeným postřikem).

\item Případně došlo k chybnému rozpoznání přesnosti odkazu \(2,5V\) pokud není pin PC4 (pin 27) připojen.
Odstranění je možné pomocí dodatečného Pull-Up odporu k VCC.

\item Funkce diody gate triaku nelze změřit.

\item Příležitostně byly hlášeny problémy s prahovou hodnotou pro vypínání \(4,3V\) pro procesory ATmega168 nebo ATmega328. To vede k resetování při měření kondenzátorů. Příčina je neznámá.
Chyba zmizí, když je práh Brown-Out nastaven na hodnotu \(2,7V\).

\item Při použití režimu spánku procesoru se spotřeba energie VCC pohybuje více než
ve starších verzích softwaru.
Pokud zjistíte nějaké problémy, měli byste zkontrolovat blokovací kondenzátory.
Keramické \(100nF\) kondenzátory by měly být připojeny v blízkosti napájecích pinů konektorů ATmega.
Můžete také zabránit tomu, aby byl stav spánku použit s volbou makefile INHIBIT\_SLEEP\_MODE.

\item Měření tantalových elektrolytických kondenzátorů často způsobuje problémy.
Mohou být rozpoznány jako diody nebo dokonce nejsou rozpoznány.
Někdy pomáhá změna polarity.

\item U JFET není někdy možné správně rozlišit Source a Drain.
Příčinou je symetrická struktura těchto polovodičů.
Tento problém lze rozpoznat, že displej zůstává stejný s určenými parametry,
když jsou spojnice obráceny.
Bohužel, vím, že neexistuje způsob, jak správně polarizovat zdroj a polarizaci součástky.
Ale výměna zdroje a polarity součástky v jakémkoli obvodu by nemělo zpravidla způsobit problém.

\end{enumerate}
