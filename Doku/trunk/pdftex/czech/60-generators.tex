\chapter{Generace signálu}
\vspace*{-1cm}
Různé režimy generování signálu mohou pracovat pouze s jedním ATmega328 procesorem.

Dále musí být také zapnuta dialogová funkce s volbou makefile WITH\_MENU.
\\Ovládací nabídku lze vyvolat dlouhém stisknutím tlačítka.

Volitelné funkce se zobrazují na druhém řádku displeje.
\\Zobrazenou funkci lze zase zvolit dlouhým stiskem tlačítka.

Další funkce bude automaticky zobrazeny po 5 vteřinách nebo po krátkém stisknutí tlačítka.
\label{sec:generation}
\vspace*{-0.6cm}
\section{Generování frekvence}
\vspace*{-0.3cm}
Generování frekvence se spustí, když je funkce menu ,,f-Generator'' spuštěna dlouhým stisknutím klávesy.

Frekvence se vyvede přes odpor \(680\Omega\) na měřící port TP2.

Měřící port TP1 se přepne na GND.
\\Frekvence jsou generovány pomocí 16 bitového čítače z CPU frekvence (\(8MHz\) nebo \(16MHz\)) taktu.

Frekvenci lze nastavit po dekádách začínajících číslicí \(1Hz\) číslicemi 0-9.

Nejvyšší volitelná číslice je číslice \(100kHz\). Zde lze nastavit čísla až na 20.

Výstupní kmitočty jsou tedy nastavitelné až do \(2MHz\).
\\Bez otočného kodéru lze hodnotu nastavit krátkým stiskem kláves (\textless~0.8s).
\\Pomocí rotačních snímačů lze hodnoty podle potřeby zvýšit nebo snížit.

Při delším stisknutí tlačítka se změní pozice v dekádě.
\\V tomto případě sloupec 1 ukazuje, v jakém směru se změní další pozice po delším stisknutí.

Pro znak \textgreater~ v 1 sloupci frekvenční linky bylo zvoleno další vyšší místo.

Pro znak \textless~ v 1 sloupci je vybrána další dolní číslice (\(1Hz\)).
\\Je-li nastavena nejvyšší číslice (\(100kHz\)), zobrazí se namísto \textgreater~ R.

Pak delší stisknutí tlačítka způsobí reset frekvence na \(1Hz\).
\\Protože nelze každou nastavenou frekvenci správně generovat, je frekvenční odchylka vygenerovaného signálu zobrazena ve 3 řádku nebo za hodnotou frekvence.
\\Dlouhým stisknutím klávesy (\textgreater~2s) se vrátí tester do dialogové funkce,
kde lze zvolit jinou funkci.
\vspace*{-0.6cm}
\section{Generování impulsů s pulzní šířkovou modulací}
\vspace*{-0.3cm}
Generování šířky impulsu je možné vybrat v menu volbou 10 bitové PWM.
\\Dlouhým stisknutím tlačítka je možné, tuto volbu nastartovat.

Frekvence je vyvedena přes \(680\Omega\) odpor na měřící port TP2.

Měřící port TP1 se přepne na GND.

Výstupní kmitočet je pevný a CPU takt je dělen 1024.

To dává frekvenci \(7812,5Hz\) při frekvenci \(8MHz\) taktu.
\\Stisknutím tlačítka  lze změnit pouze pozitivní šířku impulsu.

Každým krátkým stiskem se zvýší šířka pozitivního impulzu  o \(1\%\) až k \(99\%\).

Dlouhým stisknutím tlačítka se šířka impulsu zvyšuje o \(10\%\).
\\Když šířka impulzu dosáhne hodnoty více než \(99\%\), bude od výsledku 100 odečteno.

Šířka impulsu \(0\%\)  vytváří velmi krátký pozitivní impuls.

