\chapter{Bedienungshinweise}
\label{sec:manual}
\section{Der Messbetrieb}
Die Bedienung des Transistortesters ist einfach.
Trotzdem sind einige Hinweise erforderlich.
Meistens sind an die drei Testports über Stecker-Leitungen mit Krokodilklemmen oder anderen Klemmen angeschlossen.
Es können auch Fassungen für Transistoren angeschlossen sein.
In jedem Fall können Sie Bauteile mit drei Anschlüssen mit den drei Testports in beliebiger Reihenfolge verbinden.
Bei zweipoligen Bauteilen können Sie die beiden Anschlüsse mit beliebigen Testports verbinden.
Normalerweise spielt die Polarität keine Rolle, auch Elektrolytkondensatoren können beliebig angeschlossen werden.
Die Messung der Kapazität wird aber so durchgeführt, dass der Minuspol am Testport mit der kleineren Nummer liegt.
Da die Messspannung aber zwischen \(0,3V\) und maximal \(1,3V\) liegt, spielt auch hier die Polarität keine wichtige Rolle.
Wenn das Bauteil angeschlossen ist, sollte es während der Messung nicht berührt werden. Legen Sie es auf einen
isolierenden Untergrund ab, wenn es nicht in einem Sockel steckt. Berühren Sie auch nicht die Isolation der Messkabel,
das Messergebnis kann beeinflusst werden.
Dann sollte der Starttaster gedrückt werden.
Nach einer Startmeldung erscheint nach circa zwei Sekunden das Messergebnis. Bei einer Kondensatormessung kann es
abhängig von der Kapazität auch deutlich länger dauern.

Was dann weiter geschieht, hängt von der Softwarekonfiguration des Testers ab.
\begin{description}
  \item[Einzelmessung] Wenn der Tester für Einzelmessung konfiguriert ist (POWER\_OFF-Option),
 schaltet der Tester nach einer Anzeigezeit
von 28 Sekunden (konfigurierbar) wieder automatisch aus, um die Batterie zu schonen.
Während der Anzeigezeit kann aber auch vorzeitig eine neue Messung gestartet werden.
Nach der Abschaltung kann natürlich auch wieder eine
neue Messung gestartet werden, entweder mit dem gleichen Bauteil oder mit einem anderen Bauteil.
Wenn die Elektronik zum Abschalten fehlt, wird das letzte Messergebnis weiter angezeigt.

  \item[Dauermessung] Einen Sonderfall stellt die Konfiguration ohne die automatische Abschaltfunktion dar.
Hierfür wird die POWER\_OFF-Option in der Makefile nicht gesetzt.
Diese Konfiguration wird normalerweise nur ohne die Transistoren für die Abschaltung benutzt.
Es wird stattdessen ein externer Ein-/Aus-Schalter benötigt. Hierbei wiederholt der Tester die
Messungen solange, bis ausgeschaltet wird.

  \item[Serienmessung] In diesem Konfigurationsfall wird der Tester nicht nach einer Messung sondern erst nach einer konfigurierbaren
Zahl von Messungen abgeschaltet. Hierfür wird der POWER\_OFF-Option in der Makefile eine Wiederholzahl (z.B. 5) zugewiesen.
Im Standardfall wird der Tester nach fünf Messungen ohne erkanntes Bauteil abgeschaltet.
Wird ein angeschlossenes Bauteil erkannt, wird erst bei der doppelten Anzahl, also zehn Messungen abgeschaltet.
Eine einzige Messung mit nicht erkanntem Bauteil setzt die Zählung für erkannte Bauteile auf Null zurück.
Ebenso setzt eine einzige Messung mit erkannten Bauteil die Zählung für die nicht erkannten Bauteile auf Null zurück.
Dies hat zur Folge, dass auch ohne Betätigung des Starttasters immer weiter gemessen werden kann,
 wenn Bauteile regelmässig gewechselt werden.
Ein Bauteilwechsel führt in der Regel durch die zwischenzeitlich leeren Klemmen zu einer Messung ohne erkanntes Bauteil.

Eine Besonderheit gibt es in diesem Betriebsmodus für die Anzeigezeit. Wenn beim Einschalten der Starttaster nur kurz
gedrückt wurde, beträgt die Anzeigezeit der Messergebnisse nur 5 Sekunden. Wenn der Starttaster bis zum Erscheinen der
ersten Meldung festgehalten wurde, beträgt die Anzeigezeit wie bei der Einzelmessung 28 Sekunden.
Ein vorzeitiger neuer Messbeginn ist aber während der Anzeigezeit durch erneutes Drücken des Starttasters möglich.

\end{description}

\section{Optionale Menüfunktionen für den ATmega328}
Wenn die Menüfunktion eingeschaltet ist, startet der Tester nach einem längeren Tastendruck (\textgreater~\(0.5s\)) ein Auswahlmenü
für zusätzliche Funktionen.
Diese Funktion ist auch für andere Prozessoren mit mindestens 32K Flash Speicher verfügbar.
Die wählbaren Funktionen erscheinen in Zeile~2 eines zweizeiligen Displays oder werden als gekennzeichnete
Funktion in Zeile 3 eines vierzeiligen Displays gezeigt. Dabei wird die vorige und nächste Funktion in Zeile 2 und 4 angezeigt.
Nach einer längeren Wartezeit ohne jegliche Bedienung kehrt das Programm zu der normalen Transistortester-Funktion zurück.
Durch kurzen Tastendruck kann zur nächsten Auswahl gewechselt werden.
Mit einem längeren Tastendruck startet die angezeigte Zusatzfunktion.
Nach Anzeige der letzten Funktion \inquotes{Schalte aus} wird wieder die erste Funktion angezeigt.\\

Wenn der Tester mit einem Drehimpulsgeber ausgerüstet ist, kann die Menü-Auswahl auch mit einem schnellen
Drehen des Encoders während der Anzeige einer vorausgegangenen Messung aufgerufen werden.
Die Menüfunktionen können mit einem langsamen Drehen des Encoders in beliebiger Richtung ausgewählt werden.
Die ausgewählte Menüfunktion kann aber nur mit einem längeren Tastendruck gestartet werden.
Innerhalb einer Menüfunktion können Parameter durch eine langsame Drehung des Encoders ausgewählt werden.
Eine schnelle Drehung des Encoders kehrt zur Menü-Auswahl zurück.

\begin{description} \setlength{\itemsep}{0em}
 \item[Frequenz]
 Die Zusatzfunktion \inquotes{Frequenz} (Frequenzmessung) benutzt als Eingang den PD4-Pin des ATmega, der auch an das LCD angeschlossen ist.
Es wird immer zunächst die Frequenz gemessen, bei Frequenzen unter \(25kHz\) wird auch die mittlere Periode des Eingangssignals
bestimmt und daraus die Frequenz mit einer Auflösung von bis zu \(0,001Hz\) berechnet.
Bei gesetzter POWER\_OFF-Option in der Makefile wird die Dauer der Frequenzmessung auf 8~Minuten beschränkt.
Die Frequenzmessung wird durch Tastendruck beendet und in das Auswahlmenü zurückgekehrt.\\

 \item[f-Generator]
 Bei der Zusatzfunktion \inquotes{f-Generator} (Frequenz-Generator) können die Frequenzen zwischen 1Hz und 2MHz gewählt werden.
Die Einstellung der Frequenz kann jeweils nur für die höchste dargestellte Stelle (Ziffernposition) verändert werden.
Für die Stellen 1Hz bis 10kHz sind jeweils die Ziffern 0-9 wählbar. Bei der 100kHz Stelle ist 0-20 wählbar.
In Spalte 1 der Frequenzzeile wird durch ein \textgreater~oder \textless~Symbol angezeigt, ob durch einen längeren
Tastendruck (\textgreater~0.8s) zur höheren oder niedrigen Stelle geschaltet wird.
Zur niedrigeren Stelle (\textless) kann nur geschaltet werden, wenn die augenblickliche Stellee auf 0 gestellt wurde
und wenn nicht die Stelle mit 1Hz Schritten gewählt wurde.
Bei der gewählten 100kHz Stelle ist das \textgreater~Symbol durch ein R Zeichen ersetzt. Der längere Tastendruck bewirkt dann eine
Rücksetzung der Frequenz auf den Startwert 1Hz.
Bei gesetzter POWER\_OFF-Option in der Makefile muss für den Frequenzwechsel die Taste länger gedrückt werden, da
durch einen kurzen Tastendruck (\textless~0,2s) nur die Zeitüberwachung von 4~Minuten zurückgesetzt wird.
Die abgelaufene Zeit wird in Zeile 1 durch einen Punkt für jede abgelaufene 30 Sekunden angezeigt.
Durch einen regelmäßigen kurzen Tastendruck kann die vorzeitige Abschaltung der Frequenzerzeugung verhindert werden.
Ein längerer Tastendruck (\textgreater~2s) kehrt wieder zur Auswahl der Funktionen zurück.\\

 \item[10-bit PWM]
 Bei der Zusatzfunktion \inquotes{10-bit PWM} (Pulsweitenmodulation) wird eine feste Frequenz mit einstellbarer Pulsweite an Pin TP2 erzeugt.
Mit einem kurzen Tastendruck (\textless~0,5s) wird die Pulsweite um \(1\%\) erhöht, mit einem längeren Tastendruck um \(10\%\).
Bei Überschreiten von \(99\%\) werden \(100\%\) vom erhöhten Wert abgezogen.
Bei gesetzter POWER\_OFF-Option in der Makefile wird die Frequenzerzeugung nach 8 Minuten ohne Bedienung beendet.
Durch sehr langen Tastendruck (\textgreater~1,3s) kann die Frequenzerzeugung auch beendet werden.\\

 \item[C+ESR@TP1:3]
 Bei der Zusatzfunktion \inquotes{C+ESR@TP1:3} wird eine separate Kondensatormessung mit ESR-Messung an TP1 und TP3 gestartet.
Messbar sind Kondensatoren mit mehr als \(2\mu F\) bis zu \(50mF\). Wegen der geringen Messspannung von etwa 300mV sollte
in vielen Fällen die Messung in der Schaltung ohne vorherigen Ausbau möglich sein.
Bei gesetzter POWER\_OFF-Option in der Makefile ist die Anzahl der Messungen auf 250 beschränkt,
kann aber sofort wieder gestartet werden.
Die Mess-Serie kann durch einen längeren Tastendruck beendet werden.\\

 \item[Widerstandsmeßgerät]
Mit dem \mbox{1 \electricR 3} Symbol wird der Tester in ein Ohmmeter an TP1 und TP3 verwandelt. Diese Betriebsart wird durch
ein \textbf{[R]} in der rechten Ecke der ersten Displayzeile angezeigt. Weil bei dieser Betriebsart die
ESR-Meßfunktion nicht benutzt wird, beträgt auch für Widerstände unter \(10\Omega\) die Auflösung
nur \(0.1\Omega\). Wenn die Ohmmeter Funktion mit der Induktitivitätsmessung konfiguriert wurde,
erscheint hier ein \mbox{1 \electricR \electricL 3} Symbol.
Dann beinhaltet die Ohmmeter Funktion die Messung von Induktivitäten für Widerstände unter \(2100\Omega\). 
In der rechten Ecke der ersten Zeile des Displays wird dann ein \textbf{[RL]} angezeigt.
Für Widerstände unter \(10\Omega\) wird dann auch die ESR-Meßmethode benutzt,
wenn keine Induktivität festgestellt wurde. Damit erhöht sich die Auflösung für Widerstände
unter \(10\Omega\) auf \(0.01\Omega\).
In dieser Betriebsart werden die Meßwerte fortlaufend ermittelt.
Mit einem Tastendruck beendet der Tester diese Betriebsart und kehrt wieder zum Menü zurück.
Die gleiche Betriebsart wird auch automatisch gestartet, wenn zwischen TP1 und TP3 ein einzelner Widerstand
angeschlossen wurde und der Start-Taster gedrückt wurde. In diesen Fall kehrt der Tester mit einem Tastendruck 
wieder zu der normalen Teseterfunktion zurück.

 \item[Kondensatormeßgerät]
Mit dem \mbox{\begin{large}1 \electricC 3\end{large}} Symbol wird der Tester
in ein reines Kondensator-Meßgerät an TP1 und TP3 verwandelt.
Die Betriebsart wird durch ein \textbf{[C]} in der rechten Ecke der ersten Zeile des Displays angezeigt.
Bei dieser Betriebsart können Kondensatoren ab \(1pF\) bis zu \(100mF\) gemessen werden.
In dieser Betriebsart werden die Meßwerte fortlaufend ermittelt.
Mit einem Tastendruck wird dieser Sonderbetrieb beendet und der Tester kehrt wieder zum Menü zurück.
Auch hier wird wie bei der Widerstands-Meßfunktion in den  Betriebs-Modus automatisch gewechselt,
 wenn zwischen TP1 und TP3 ein Kondensator mit der normalen Testerfunktion gemessen wurde.
Der Tester kehrt nach dem automatischen Start wieder zu der normalen Testerfunktion zurück, wenn der Taster gedrückt wird.

 \item[Impulsdrehgeber]
 Mit der Zusatzfunktion \inquotes{Impulsdrehgeber} kann ein Drehgeber untersucht werden.
Die drei Kontakte des Impulsdrehgebers müssen vor dem Start der Zusatzfunktion beliebig an die drei Testpins
 des Transistortesters angeschlossen werden.
Nach dem Start der Funktion muss der Drehknopf nicht zu schnell gedreht werden.
Wenn der Test erfolgreich abgeschlossen ist, wird die Pinbelegung der Kontakte symbolisch in Zeile 2 dargestellt.
Dabei wird der gemeinsame Anschluss herausgefunden und für etwa zwei Sekunden angezeigt,
ob an den Raststellung beide Kontake offen ('o') oder beide Kontakte geschlossen ('C') sind.
Ein Impulsdrehgeber mit offenen Kontakten an den Raststellungen wird so mit der Zeile 2 \inquotes{1-/-2-/-3~o} zwei Sekunden lang angezeigt.
Natürlich wird die richtige Pinnummer des gemeinsamen Kontaktes in der Mitte anstelle der '2' angezeigt.
Wenn auch die geschlossene Schalterstellung an den Raststellungen vorkommt,
wird außerdem in Zeile 2 \inquotes{1---2---3~C} zwei Sekunden lang angezeigt.
Mir ist kein Impulsdrehgeber bekannt, der immer nur geschlossene Kontakte an jeder Raststellung hat.
Die Stellungen der Kontakte zwischen den Raststellungen werden nur kurz (\textless\(~0,5s\)) ohne die Kennbuchstaben 'o' oder 'C' 
in Zeile 2 angezeigt.
Wenn der Impulsdrehgeber für die Bedienung des Testers eingesetzt werden soll, muß die Makefile Option WITH\_ROTARY\_SWITCH=2
für Drehgeber mit nur den offenen Kontakten ('o') und die Option WITH\_ROTARY\_SWITCH=1 für Drehgeber mit
offenen ('o') und geschlossenen ('C') Kontakten an den Raststellungen.\\

\item[C(\(\mu F\))-Korrektur]
Mit dieser Menüfunktion kann ein Korrekturwert für die Messung grösserer Kapazitätswerte geändert werden.
Die Funktion ist die gleiche, die auch mit der Makefile Option C\_H\_KORR voreingestellt werden kann.
Werte über Null reduzieren den Ausgabewert der Kapazität um diesen Prozent Wert, Werte unter Null würden den Ausgabewert
anwachsen lassen. Ein kurzer Tastendruck verringert den Korrekturwert um 0.1\%, ein längerer Tastendruck vergrößert
den Korrekturwert um 0.1\%. Ein sehr langer Tastendruck speichert den Wert.  Es ist eine Eigenschaft des
Meßverfahrens, daß Kondensatoren mit geringer Güte wie Elektrolyt-Kondensatoren mit zu großer Kapazität gemessen werden.
Erkennbar ist die Güte über den Parameter Vloss. Gute Kondensatoren haben kein Vloss oder nur 0.1\%.
Zum Abgleich dieses Parameters sollten also nur Kondensatoren mit über \(50\mu F\) hoher Güte verwendet werden.
Übrigens halte ich die Bestimmung des exakten Kapazitätswertes von Elektrolyt-Kondensatoren für überflüssig,
da die Kapazität sowohl von der Temperatur als auch der DC-Spannung abhängt.

 \item[Selbsttest]
 Mit der Zusatzfunktion \inquotes{Selbsttest} wird ein vollständiger Selbsttest mit Kalibration durchgeführt.
Dabei werden sowohl die Testfunktionen T1 bis T7 (wenn nicht verhindert mit der Option NO\_TEST\_T1\_T7) 
als auch die Kalibration mit dem externen Kondensator jedes Mal durchgeführt.\\

 \item[Spannung]
 Die Zusatzfunktion \inquotes{Spannung} (Spannungsmessung) ist nur möglich, wenn die serielle Ausgabe deaktiviert wurde
oder der ATmega mindestens 32 Pinne hat (PLCC) und einer der zusätzlichen Pinne ADC6 oder ADC7 für die Messung benutzt wird.
Da am Port PC3 (oder ADC6/7) ein 10:1-Spannungsteiler vorgesehen ist, können Spannungen bis \(50V\) gemessen werden.
Ein installierter DC-DC-Wandler für die Zenerdioden-Messung wird durch Tastendruck eingeschaltet.
So können auch angeschlossene Zenerdioden gemessen werden.
Bei gesetzter POWER\_OFF-Option in der Makefile und ohne Bedienung wird die Messung nach 4~Minuten beendet.
Die Messung kann aber durch einen besonders langen Tastendruck (\textgreater~4~Sekunden) vorher beendet werden.\\

\item[Kontrast] 
Diese Funktion steht für Controller mit Kontrasteinstellung per Software zur Verfügung.
Der Einstellwert kann mit einem sehr kurzen Tastendruck oder einer Linksdrehung des Impulsdrehgebers verringert werden.
Ein längerer Tastendruck oder eine Rechtsdrehung des Impulsdrehgebers vergrößert den Einstellwert.
Die Funktion wird beendet und der eingestellte Wert wird permanent in den EEPROM geschrieben, 
wenn die Taste noch länger gedrückt wird.

\item[BackColor]
Für Farbdisplays kann dieser Menüpunkt mit der Makefile Option LCD\_CHANGE\_COLOR eingeschaltet werden,
um die Hintergrundfarbe einstellen zu können. Die Impulsdrehgeber-Erweiterung muß dafür installiert sein.
Sie können eine der drei Farben rot, grün und blau mit einem längeren Tastendruck wählen.
Die Intensität der gewählten Farbe, die mit einem {\textgreater} Zeichen in Spalte 1 gekennzeichnet wird,
kann durch Drehen des Impulsdrehgebers verändert werden.

\item[FrontColor]
Für Farbdisplays kann dieser Menüpunkt mit der Makefile Option LCD\_CHANGE\_COLOR eingeschaltet werden,
um die Vordergrundfarbe einstellen zu können. Die Impulsdrehgeber-Erweiterung muß dafür installiert sein.
Sie können eine der drei Farben rot, grün und blau mit einem längeren Tastendruck wählen.
Die Intensität der gewählten Farbe, die mit einem {\textgreater} Zeichen in Spalte 1 gekennzeichnet wird,
kann durch Drehen des Impulsdrehgebers verändert werden.

 \item[Zeige Daten]
 Die Funktion \inquotes{Zeige Daten} zeigt neben der Software-Versionsnummer die Daten des Abgleichs an.
Dies sind die Nullwiderstände R0 von Pin 1:3, 2:3 und 1:2.
Außerdem werden die Ausgangswiderstände der Ports zur \(5V\)-Seite (RiHi) und 
zur \(0V\) Seite (RiLo) angezeigt.
Die Nullkapazitätswerte (C0) werden ebenfalls in allen Pinkombinationen angezeigt (1:3, 2:3, 1:2 und 3:1, 3:2 2:1).
Danach werden auch die Spannungskorrekturen für die Komparatorspannung (REF\_C) und für die Referenzspannung (REF\_R) angezeigt.
Bei Graphikdisplays werden danach noch die verwendeten Symbole für die Bauteile und der Fontsatz angezeigt.
Jede Seite wird 15 Sekunden angezeigt.
Es kann aber auch durch Tastendruck oder einer Rechtsdrehung des Impulsdrehgebers
zur nächsten Seite geblättert werden.
Mit einer Linksdrehung des Impulsdrehgebers kann die Ausgabe wiederholt werden oder zur vorigen Seite zurückgeblättert werden.

 \item[Schalte aus]
 Mit der Zusatzfunktion \inquotes{Schalte aus} kann der Transistortester abgeschaltet werden.\\

 \item[Transistor]
 Natürlich kann mit der Funktion \inquotes{Transistor} (Transistortester) wieder zu der normalen Transistortester-Funktion
zurückgekehrt werden.

\end{description}

Bei gesetzter POWER\_OFF-Option in der Makefile sind alle Zusatzfunktionen zeitbeschränkt, damit die Batterie nicht verbraucht wird.

\section{Selbsttest und Kalibration}

Wenn die Software mit der Selbsttestfunktion konfiguriert ist, kann der Selbsttest durch einen Kurzschluss aller drei
Testports und Drücken der Starttaste eingeleitet werden.
Für den Start des Selbsttests muss die Starttaste innerhalb von 2 Sekunden noch einmal gedrückt werden,
sonst wird mit einer normalen Messung fortgefahren.
Beim Selbsttest werden die im Selbsttest-Kapitel~\ref{sec:selftest} beschriebenen Tests ausgeführt.
Wenn der Tester mit einer Menüfunktion (Option WITH\_MENU) konfiguriert ist, 
wird der vollständige Selbsttest mit den Tests T1 bis T7 nur bei dem \inquotes{Selbsttest} ausgeführt, 
der als Menüfunktion ausgewählt werden kann.
Außerdem wird beim Aufruf über die Menüfunktion der Abgleich mit dem externen Kondensator jedesmal durchgeführt,
sonst nur für den ersten Aufruf.
Damit kann der durch die kurgeschlossenen Testports automatisch gestartete Abgleich schneller durchgeführt werden. 
Die viermalige Testwiederholung bei T1 bis T7 kann vermieden werden, wenn der Starttaster gedrückt gehalten wird.
So kann man uninteressante Tests schnell beenden und
sich durch Loslassen des Starttasters interessante Tests viermal wiederholen lassen.
Der Test 4 läuft nur automatisch weiter, wenn die Verbindung zwischen den Testports gelöst wird.

Wenn die Funktion AUTO\_CAL in der Makefile gewählt ist, wird beim Selbsttest
eine Kalibration der Nullwertes für die Kondensatormessung durchgeführt.
Für die Kalibration des Nullwertes für die Kondensatormessung ist wichtig, 
dass die Verbindung zwischen den Testpins (Kurzschluß) während des Tests 4 wieder gelöst wird!
Sie sollten während der Kalibration (nach dem Test 6) weder die Testports noch angeschlossene Kabel berühren.
Die Ausrüstung sollte aber die gleiche sein, die später zum Messen verwendet wird.
Anderenfalls wird der Nullwert der Kondensatormessung nicht richtig bestimmt.
Die Kalibration des Innenwiderstandes der Port-Ausgänge wird mit dieser Option vor jeder
Messung durchgeführt. \\

Bei der Kalibration werden zwei Sonderschritte gemacht, wenn die samplingADC Funktion in der Makefile gewählt wurde (WITH\_SamplingADC = 1).
Nach der normalen Bestimmung der Nullwerte der Kapazitätsmessung werden dann auch die Nullwerte mit
der Sampling Methode (C0samp) bestimmt. Als letzter Teil der Kalibration wird der Anschluß eines Testkondensators für die
Spulenmessung an Pin~1 und Pin~3 angefordert mit der Meldung \mbox{\begin{large}1 \electricC 3~10-30nF(L)\end{large}}.
Der Kapazitätswert sollte hierbei zwischen \(10nF\) und \(30nF\) liegen,
um eine meßbare Resonanzfrequenz bei der späteren Parallelschaltung mit einer Spule (\textless~\(2mH\)) zu erreichen.
Für Spulen mit mehr als 2mH Induktivität sollte die normale Testfunktion ohne die Parallelschaltung eines
Kondensators ausreichen. Ein Parallelschalten des Kondensators sollte hier nicht mehr 
zu einer Verbesserung des Meßergebnisses führen.\\

Nach der Bestimmung der Nullkapazitäten ist der Anschluss eines Kondensators 
mit einer beliebiger Kapazität zwischen \(100nF\) und \(20\mu F\) an Pin~1 und Pin~3 erforderlich.
Dazu wird in Zeile 1 ein \mbox{\begin{large}1 \electricC 3~\textgreater 100nF\end{large}} angezeigt.
Sie sollten den Kondensator erst nach der Ausgabe der C0 Werte oder nach der Ausgabe dieser Aufforderung anschließen.
Mit diesem Kondensator wird die Offset-Spannung des analogen Komparators kompensiert,
um genauere Kapazitätswerte ermitteln zu können.
Die Verstärkung für ADC-Messungen mit der internen Referenz-Spannung wird ebenfalls mit diesem Kondensator abgeglichen, um
bessere Widerstands-Messergebnisse mit der AUTOSCALE\_ADC-Option zu erreichen.
Wenn die Menüfunktion beim Tester ausgewählt wurde (Option WITH\_MENU) und der Selbsttest
nicht als Menüfunktion gestarte wurde, wird der Abgleich mit dem externen Kondensator nur bei
der ersten Kalibration durchgeführt.
Die Kalibration mit dem externen Kondensator kann nur wiederholt werden, wenn der Selbsttest
als Menüfunktion ausgewählt wird.

Der Nullwert für die ESR-Messung wird als Option ESR\_ZERO in der Makefile vorbesetzt.
Mit jedem Selbsttest wird der ESR Nullwert für alle drei Pinkombinationen neu bestimmt.
Das Verfahren der ESR-Messung wird auch für Widerstände mit Werten unter \(10\Omega\) benutzt um
hier eine Auflösung von \(0,01\Omega\) zu erreichen.

\section{Besondere Benutzungshinweise}
Normalerweise wird beim Start des Testers die Batteriespannung angezeigt. Wenn die Spannung eine Grenze unterschreitet, 
wird eine Warnung hinter der Batteriespannung ausgegeben. Wenn Sie eine aufladbare \(9V\)-Batterie benutzen, sollten Sie
den Akku möglichst bald austauschen oder nachladen.
Wenn Sie eine Version mit eingebauter \(2,5V\)-Präzisionsreferenz besitzen, wird beim Start für eine Sekunde die
gemessene Betriebsspannung in der Zeile 2 mit \inquotes{VCC=x.xxV} angezeigt.

Es kann nicht oft genug erwähnt werden, dass Kondensatoren vor dem Messen entladen sein müssen.
Sonst kann der Tester schon defekt sein, bevor der Startknopf gedrückt ist.
Wenn man versucht, Bauelemente im eingebauten Zustand zu messen, sollte das Gerät immer von
der Stromquelle getrennt sein. Außerdem sollte man sicher sein, dass keine Restspannung mehr
im Gerät vorhanden ist. Alle Geräte haben Kondensatoren verbaut!

Beim Messen kleiner Widerstandswerte muss man besonders auf die Übergangswiderstände achten.
Es spielt die Qualität und der Zustand von Steckverbindern eine große Rolle, genau so wie die
Widerstandwerte von Messkabeln. Dasselbe gilt auch für die Messung des ESR-Wertes von Kondensatoren.
Bei schlechten Anschlusskabeln mit Krokodilklemmen wird so aus einem ESR von \(0,02\Omega\) leicht
ein Wert von \(0,61\Omega\).
Wenn möglich sollte man Kabel mit Testklemmen an die drei Testports parallel zu vorhandenen Sockeln fest anschließen (anlöten).
Dann braucht der Tester für kleine Kapazitäten nicht jedesmal neu kalibriert werden, wenn mit oder ohne eingesteckte
Testkabel gemessen wird. Für die Kalibration des Nullwiderstandes macht es aber im allgemeinen einen Unterschied, 
ob die Testpins direkt am Sockel oder am Ende der Kabel mit den den Testklemmen verbunden wird. Nur im letzteren
Fall ist der Widerstand von Kabel und Klemmen mit kalibriert. Im Zweifelsfall kann man die Kalibration mit dem
Kurzschluß am Testsockel durchführen und danach den Widerstand der kurzgeschlossenen Klemmen mit dem Tester messen.

An die Genauigkeit der Messwerte sollte man keine übertriebenen Erwartungen haben, dies gilt besonders
für die ESR Messung und die Induktivitätsmessung.
Die Ergebnisse meiner Messreihen kann man im Kapitel~\ref{sec:measurement} ab Seite~\pageref{sec:measurement} finden.



\section{Problemfälle}
Bei den Messergebnissen sollten Sie immer im Gedächtnis behalten, dass die Schaltung des Transistortesters für
Kleinsignal-Bauelemente ausgelegt ist. Normalerweise beträgt der maximale Messstrom etwa \(6mA\).
Leistungshalbleiter machen oft wegen hoher Reststöme Probleme bei der Erkennung oder beim Messen der
Sperrschicht-Kapazität.
Bei Thyristoren und Triacs werden oft die Zündströme oder die Halteströme nicht erreicht. Deswegen kann es
vorkommen, dass ein Thyristor als NPN-Transistor oder Diode erkannt wird. Ebenso ist es möglich, dass ein 
Thyristor oder Triac gar nicht erkannt wird.

Probleme bei der Erkennung machen auch Halbleiter mit integrierten Widerständen.
So wird die Basis-Emitter-Diode eines BU508D-Transistors wegen eines parallel geschalteten internen
\(42 \Omega\) Widerstandes nicht erkannt.
Folglich kann auch die Transistorfunktion nicht geprüft werden.
Probleme bei der Erkennung machen oft auch Darlington-Transistoren höherer Leistung. Hier sind auch
oft Basis-Emitter-Widerstände verbaut, welche die Erkennung wegen der hier verwendeten kleinen Messströme erschweren.

\section{Messung von PNP- und NPN-Transistoren}
Normalerweise werden die drei Anschlüsse des Transistors in beliebiger Reihenfolge an die Messeingänge des
Transistortesters angeschlossen.
Nach dem Drücken des Starttasters meldet der Tester in der Zeile 1 den Typ (NPN oder PNP), 
eine eventuell vorhandene Schutzdiode der Kollektor-Emitter-Strecke und
die Anschlussbelegung. Das Diodensymbol wird  polungsrichtig angezeigt.
In der Zeile 2 wird der Stromverstärkungsfaktor \(B\) oder \(hFE\) angegeben und der Strom, bei der die Verstärkung
bestimmt wurde. Wenn die Emitterschaltung für die Bestimmung des Verstärkungsfaktors verwendet wurde,
wird der Kollektorstrom \(Ic\) genannt. Bei Bestimmung des Verstärkungsfaktors in Kollektorschaltung (Emitterfolger)
wird der Emitterstrom \(Ie\) angegeben. Weitere Parameter werden bei einem zweizeiligen Display
nacheinander in Zeile 2 angegeben. Bei Displays mit mehr Zeilen werden weitere Zeilen benutzt bis
die letzte Zeile schon benutzt war.
Erst bei der letzten Zeile erfolgen weitere Ausgaben nacheinander in der letzten Zeile.
Wenn weitere Parameter vorhanden sind, wird am Ende der letzten Zeile ein + Zeichen ausgegeben.
Der nächste Wert erscheint auf Knopfdruck oder nach einer Wartezeit auch automatisch.
Der nächste ausgegebene Parameter ist jedenfalls die Basis-Emitter-Schwellspannung \(Ube\),
bei der der Verstärkungsfaktor bestimmt wurde.
Falls meßbar, werden auch der Kollektor-Reststrom mit offener Basis \(I_{CE0}\) und mit kurzgeschlossener
Basis \(I_{CES}\) ausgegeben.
Falls eine Schutzdiode vorhanden ist, wird die Flußspannung \(Uf\) dieser Schutzdiode als letzter Parameter ausgegeben.


Bei der Emitterschaltung hat der Tester nur zwei Möglichkeiten, den Basisstrom einzustellen:
\begin{enumerate}
\item Mit dem \(680\Omega\) Widerstand ergibt sich ein Basisstrom von etwa \(6,1mA\). Das ist für einen
Kleinsignaltransistor mit hohem Verstärkungsfaktor meist zu viel, weil die Basis gesättigt ist.
Da der Kollektorstrom ebenfalls mit einem \(680\Omega\) Widerstand gemessen wird, kann der Kollektorstrom
den um den Verstärkungsfaktor höheren Strom gar nicht erreichen. Die Softwareversion vom Markus F. hat
in diesem Zustand die Basis-Emitter Spannung gemessen (Uf=\ldots).\\
\item Mit dem \(470k\Omega\) Widerstand] ergibt sich ein Basisstrom von nur \(9,2\mu A\) .
Das ist für einen Leistungstransistor mit geringem Verstärkungsfaktor sehr wenig.
Die Softwareversion von Markus F. hat in diesem Zustand den Stromverstärkungsfaktor bestimmt (hFE=\ldots).\\
\end{enumerate}

Die Software des Testers bestimmt die Stromverstärkung jetzt auch in der Kollektorschaltung.
Ausgegeben wird der höhere Wert der beiden Mess-Methoden.
Die Kollektorschaltung hat den Vorteil, dass sich durch die Stromgegenkopplung der Basisstrom abhängig vom
Verstärkungsfaktor reduziert. Dadurch kann für Leistungstransistoren mit dem \(680\Omega\) und für Darlington-Transistoren
mit dem \(470k\Omega\) Widerstand meist ein günstigerer Messstrom ergeben.
Die ausgegebene Basis-Emitter-Schwellspannung \(Ube\) ist jetzt die Spannung,
die bei der Bestimmung des Stromverstärkungsfaktors gemessen wurde.
Wenn man trotzdem eine Basis-Emitter-Schwellspannung bei circa \(6mA\) ermitteln möchte, muss man den Kollektor
vom Tester trennen und noch einmal messen.
Dann wird die Schwellspannung bei etwa \(6mA\) ausgegeben und die Kapazität der Diode in Sperr-Richtung ermittelt.
Natürlich kann so auch die Basis-Kollektor-Diode gemessen werden.

Bei Germanium-Transistoren wird meistens ein Kollektor-Emitter-Reststrom \(I_{CE0}\) mit stromloser Basis oder
ein Kollektor-Emitter-Reststrom \(I_{CES}\) mit Basis auf Emitterpotential gemessen.
Durch Kühlen kann der Reststrom bei Germanium-Transistoren erheblich gesenkt werden. 

\section{Messung von JFET- und D-MOS-Transistoren}
Wegen des symmetrischen Aufbaus von JFET-Transistoren kann Source und Drain nicht unterschieden werden.
Normalerweise wird bei JFETs als eine Kenngröße der Strom bei kurzgeschlossenem Gate-Source angegeben.
Dieser Strom ist aber oft höher als der, der sich bei der Messschaltung mit dem \(680\Omega\) Widerstand erreichen läßt.
Deswegen wird der \(680\Omega\) Widerstand an den Source-Anschluss geschaltet. Dadurch erhält das Gate
stromabhängig eine negative Vorspannung. Als Kenngröße wird sowohl der ermittelte Strom als auch die
Gate-Source-Spannung ausgegeben. 
Damit können verschiedene Typen unterschieden werden.
Für D-MOS-Transistoren (Verarmungs-Typ) wird das gleiche Messverfahren verwendet.

\section{Messung von E-MOS Transistoren und IGBTs}
Für Anreicherungs-MOS-Transistoren (P-E-MOS oder N-E-MOS) sollte man wissen, dass die Bestimmung der Gate-Schwellspannung (Vth)
bei kleiner Gatekapazität schwierig wird. Hier können mit dem Tester genauere Spannungswerte ermittelt werden, wenn ein
Kondensator mit einigen nF parallel zum Gate / Source angeschlossen wird.
Die Schwellspannung wird bei Drain-Strömen von etwa \(3,6mA\) für P-E-MOS und bei etwa \(4mA\) für N-E-MOS bestimmt.
Der Widerstand RDS or besser R\textsubscript{DSon} von E-MOS Transistoren wird mit einer Gate - Source Spannung von annähernd \(5V\) bestimmt,
 was wahrscheinlich nicht der niedrigste Wert ist.
Außerdem wird der RDS Widerstand bei niedrigem Drain-Strom ermittelt, was die Auflösung des Widerstandswertes begrenzt.
Oft bei IGBTs und manchmal auch bei Anreicherungs-MOS Transistoren reichen die verfügbaren \(5V\) des Testers nicht aus, 
um den Transistor über das Gate anzusteuern.
In diesem Fall hilft eine Batterie mit etwa \(3V\) aus, um eine Erkennung und Messungen mit dem Tester möglich zu machen.
Die Batterie wird dazu mit einem Pol an das Gate des Transistors angeschlossen und der andere Pol der Batterie
 wird dann anstelle des Transistor-Gates an einen Testport (TP) des Testers angeschlossen.
Bei richtiger Polung der Batterie  addiert sich die Batteriespannung zu der Steuerspannung des Testers und die 
Erkennung des Transistors gelingt.
Natürlich muß dann die Batteriespannung zu der angezeigten Gate-Schwellspannung addiert werden,
um die richtige Schwellspannung für dieses Bauteil zu erhalten.

\section{Messen von Kondensatoren}
Die Kapazitätswerte werden immer aus einer Zeitkonstanten berechnet, die sich aus der Serienschaltung von eingebauten
Widerständen mit dem Kondensator bei Ladevorgängen ergeben. Bei kleineren Kondensatoren werden die \(470k\Omega\) Widerstände
für die Messung benutzt und die Zeit bis zum Erreichen einer Schwellspannung gemessen.
Bei größeren Kondensatoren mit einigen \(10\mu F\) wird die Spannungserhöhung des Kondensators nach Ladepulsen mit den \(680\Omega\) Widerständen 
beobachtet und daraus die Kapazität berechnet. Ganz kleine Kapazitäten können mit der samplingADC Methode gemessen werden.
Dabei wird ein Aufladevorgang immer wieder erzeugt und durch Verschiebung der ADC S\&H Zeit mit dem Zeitabstand abgetastet,
der sich aus der Taktfrequenz des Prozessors ergibt. Eine komplette ADC Wandlung braucht hingegen 1664 Prozessortakte!
Bis zu 250 ADC-Werte werden so ermittelt und aus dem Spannungsverlauf die Kapazität berechnet.
Wenn die samplingADC Funktion in der Makefile gewählt wurde, werden alle Kondensatoren \textless~\(100pF\) in der 
Kondensatormeßfunktion  \textbf{[C]} mit der samplingADC Methode gemessen. Die Auflösung beträgt hier bis zu \(0.01pF\) bei
\(16MHz\) Taktfrequenz. Der Abgleich der Nullkapazität stellt bei dieser Auflösung eine besondere Herausforderung dar.
Die samplingADC Methode des Kapazitätsbestimmung wurde immer angewendet, wenn es Bruchteile von \(1pF\) dargestellt werden.
Übrigens können auch die Sperrschichtkapazitäten von Einzeldioden mit dieser Methode gemessen werden. Da die Methode
die Kapazität sowohl beim Laden als auch beim Entladen messen kann, werden zwei Kapazitätswerte angegeben. Wegen
des Kapazitätsdioden-Effektes unterscheiden sich die beiden Werte.

\section{Messen von Spulen}
Das angewandte Verfahren zum Messen der Induktivität besteht aus einer Zeitkonstantenbestimmung des Stromanstiegs.
Die Detektionsgrenze ist hier \(0.01mH\), wenn der Widerstandswert der Spule kleiner \(24\Omega\) beträgt.
Bei größeren Widerstandswerten beträgt die Auflösung nur \(0.1mH\).
Bei Widerstandswerten über \(2.1k\Omega\) kann die Methode gar nicht angewendet werden. 
Die Ausgabewerte der normalen Messung erscheinen in der zweiten Zeile (Widerstand und Induktivität).
Mit dem samplingADC Verfahren können bei größeren Spulenwerten Eigenresonanzen festgestellt werden.
Wenn diese deutlich genug ist, wird die Frequenz und die Güte Q zusätzlich zu der normalen Messung in Zeile 3 ausgegeben.

Daneben läßt sich diese Methode der Schwingfrequenz-Messung auch für die Bestimmung des Induktivitätswertes benutzen,
wenn ein ausreichend großer Kondensator mit bekanntem Kapazitätswert einer Spule mit kleiner Induktivität (\textless~\(2mH\))
parallelgeschaltet wird. Durch die Parallelschaltung des Kondensators funktioniert die normale Messung der
Zeitkonstante nicht mehr. Die Induktivität der normalen Messung wird gar nicht ausgegeben und der
Widerstandswert wird in der ersten Zeile ausgegeben, wenn die Resonanzfrequenz eine Parallelschaltung
des Kondensators erwarten lässt. Auch für diesen Fall wird eine Güte Q aus den Schwingverhalten berechnet.
Für diesen Fall wird die berechnete Induktivität an erster Stelle in Zeile 2 angegeben. Dahinter steht dann
der Text " if " gefolgt von dem Kapazitätswert des Parallelkondensators. Der Wert dieses Kondensators kann derzeit
nur in der Kalibrationsfunktion durch Messung festgelegt werden (\mbox{\begin{large}1 \electricC 3~10-30nF(L)\end{large}}) .
Die dritte Zeile zeigt wieder die Schwingfrequenz und die ermittelte Güte Q. 
Bei zweizeiligen Displays wird der Inhalt der 3. Zeile zeitverzögert in Zeile 2 ausgegeben.

