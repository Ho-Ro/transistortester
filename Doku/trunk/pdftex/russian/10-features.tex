%\newpage
\chapter{Характеристики}
\label{sec:features}
\begin{enumerate}
\item Работает с микроконтроллерами ATmega8, ATmega168 или ATmega328. Также возможно использовать ATmega644, ATmega1284, ATmega1280 или ATmega2560.
\item Отображение результатов на символьном LCD-дисплее 2x16 или 4x20.
Если используется микроконтроллер с объемом флэш-памяти, минимум 32k, то также можно применить графический дисплей 128x64
пикселя с контроллером ST7565 или SSD1306.
При этом 4-проводной интерфейс SPI или I\textsuperscript{2}C шина должны быть подключены вместо 4-битного параллельного интерфейса для
SSD1306 контроллера. Для контроллеров NT7108 или KS0108 Вы должны использовать преобразователь последовательно-параллельного
интерфейса 74HC(T)164 или 74HC(T)595.
Дисплей с контроллером PCF8812 или PCF8814 может быть использован только без больших иконок для транзисторов, так как размер дисплея 
102x65 или 96x65 пикселей недостаточен.
\item Запуск - однократное нажатие кнопки \textbf {TEST} с автоотключением.
\item Возможна работа от автономного источника, т.к. ток потребления в выключенном состоянии не превышает \(20~nA\).
\item Чтобы уменьшить ток потребления в режиме ожидания измерения, программное обеспечение, начиная с версии 1.05k, 
использует режим сна (Sleep Mode) для микроконтроллеров Atmega168 или ATmega328.
\item Автоматическое определение N-P-N и P-N-P биполярных транзисторов, N- и P-канальных MOSFET транзисторов, 
JFET транзисторов, диодов, двойных диодов, тиристоров и симисторов.
Для тиристоров и симисторов уровень открытия должен быть досягаем для тестера.
Для IGBT транзисторов сигнал \(5~V\) должен быть достаточным для открытия транзистора.  
\item Автоматическое определение расположения выводов элемента.
\item Измерение коэффициента усиления и порогового напряжения база эмиттер биполярного транзистора.
\item Транзисторы Дарлингтона идентифицируются по пороговому напряжению и коэффициенту усиления.
\item Обнаружение защитного диода в биполярных и MOSFET транзисторах.
\item Измерение порогового напряжения затвора, значение емкости затвора и R\textsubscript{DSon} до напряжение 
затвора около \(5~V\) в транзисторах MOSFET.
\item Измерение порогового напряжения затвора и величины ёмкости затвора MOSFET. 
\item Измерение одного или двух резисторов с изображением  \mbox{\electricR}
символа резистора и точностью до 4 десятичных цифр. Все символы пронумерованы соответственно номерам 
щупов Тестера (1-2-3). Таким образом, потенциометр также может быть измерен. 
\item Разрешение измерения сопротивления до  \(0,01~\Omega\), а величина измерения - до  \(50~M\Omega\).
\item Определение и измерение одного конденсатора с изображением символа конденсатора  \mbox{\electricC}
Определение и измерение одного конденсатора с изображением символа конденсатора и точностью до четырех десятичных цифр. 
Ёмкость конденсатора может быть замерена от \(25~pF\) (\(8~MHz\), \(50~pF\) – \(1~MHz\)) до \(100~mF\). 
Разрешение измерения составляет \(1~pF\) (\(8~MHz\)).
\item ESR конденсатора измеряется с разрешением \(0,01~\Omega\) для конденсаторов ёмкостью более \(20~nF\) 
и отображается числом с двумя значащими десятичными цифрами. Это возможно только для ATmega168 или ATmega328.
\item Для конденсаторов ёмкостью выше \(5000~pF\) может быть определена потеря напряжения после воздействия импульса 
зарядки. Потеря напряжения дает оценку добротности (качества) конденсатора.
\item Определение до двух диодов с изображением их символов \mbox{\electricDAK} или \mbox{\electricDKA}
в правильном порядке. 
Дополнительно отображается прямое падение напряжения на диоде.
\item Светодиод (LED) определяется как диод с прямым напряжением выше, чем у обычного диода. Два светодиода 
в одном 3-х выводном корпусе также определяются, как два диода.
\item Стабилитроны могут быть определены, если их обратное напряжение пробоя ниже \(4,5~V\). Они отображаются, 
как два диода, и могут быть идентифицированы, как стабилитроны, только по напряжению. Номера выводов, соответствующие 
символу диода, в этом случае, идентичны. Реальный вывод анода диода можно идентифицировать только по падению напряжения 
(около \(700~mV\))!
\item Если определяется более чем 3 диода, число диодов отображается дополнительно с сообщением о том, что элемент 
повреждён. Это может произойти, только если диоды присоединены ко всем трем выводам, и, по крайней мере, один из 
диодов - стабилитрон. В этом случае необходимо произвести измерения, подсоединив к двум щупам Тестера сначала 
одну пару из трех выводов элемента, затем – любую другую пару выводов элемента. 
\item Измерение величины ёмкости одиночного диода в обратном направлении. Биполярный транзистор может также быть 
проанализирован, если подключить базу и коллектор или базу и эмиттер.
Если используется ATmega с объемом флэш-памяти более 8K, то измеряется обратный ток диода с разрешением \(2~nA\).
Значение выводится на дисплей, если оно больше нуля. 
\item Одним измерением можно определить назначение выводов выпрямительного моста.
\item Конденсаторы ёмкостью ниже \(25~pF\) обычно не определяются, но могут быть измерены вместе с параллельным диодом 
или параллельным конденсатором, ёмкостью более \(25~pF\). В этом случае из результата измерения необходимо вычесть 
ёмкость подключенного параллельно элемента.
Для контроллеров, имеющих по крайней мере 32K флэш-памяти осуществляется автоматический переход на циклический
режим измерения, если конденсатор с более чем \(25~pF\) подключен к ТР1 и ТР3. В этом режиме измерения конденсаторов
Вы можете измерить ёмкость конденсаторов ниже \(25~pF\) в ТР1 и ТР3 напрямую.
\item Для резисторов сопротивлением ниже \(2100~\Omega\), если ATmega с объемом более чем 16K флэш-памяти, измеряется индуктивность. 
Иконка индуктивности \mbox{\electricL} будет показана за иконкой \mbox{\electricR}.
Диапазон измерений от  \(0,01~mH\) до \(20~H\), но точность не высока. Получить результат измерения можно только с 
единственным подключенным элементом.
\item Время тестирования большинства элементов составляет приблизительно 2 секунды. Измерение ёмкости или индуктивности 
могут увеличить время тестирования.
\item Программное обеспечение может конфигурироваться, чтобы произвести ряд измерений прежде, чем питание будет 
отключено.
\item В функции самопроверки встроен дополнительный генератор частоты на \(50~Hz\), чтобы проверить точность тактовой 
частоты (только для контроллеров с объемом более чем 32K флэш-памяти).
\item Подключаемое, в режиме самопроверки, оборудование для тарировки внутреннего выходного сопротивления порта и 
смещения нуля при измерении ёмкости (только для контроллеров с объемом более чем 16K флэш-памяти). 
Для тарировки необходимо подключить к щупам 1 и 3 
внешний высококачественный конденсатор ёмкостью между \(100~nF\) и \(20~\mu F\) чтобы измерить величину компенсации 
напряжения смещения аналогового компаратора. Это уменьшит ошибки измерения ёмкости конденсаторов до \(40~\mu F\).
Этот же конденсатор применяется при коррекции напряжения внутреннего ИОН, замеренного для подстройки масштаба АЦП при 
измерении с внутренним ИОН.
\item Отображение обратного тока коллектора \(I_{CE0}\) при отключенной базе (с разрешением \(1~\mu A\)) и
обратного тока коллектора при короткозамкнутых выводах базы и эмиттера \(I_{CES}\) 
(только для ATmega с объемом флэш-памяти больше 16K).
Эти значения отображаются если они не равны нулю (главным образом, для германиевых транзисторов).
\item Для ATmega с объемом флэш-памяти не менее 32K тестер запускает циклический тест сопротивления резисторов,
как только резистор будет обнаружен в тестовых контактах 1 (TP1) и 3 (TP3).
Если вы выбрали дополнительное измерение индуктивности для функции циклического теста резисторов в
Makefile, задав параметр RMETER\_WITH\_L, то индуктивность также может быть определена и измерена в этом режиме.
Этот циклический режим работы обозначается \textbf{[R]} или \textbf{[RL]} справа в первой строке экрана дисплея.
Таким же образом запускается циклическое измерение ёмкости, если конденсатор, обнаружен в TP1 и TP3.
Этот режим работы отображается символом \textbf{[C]} справа в первой строке.
В циклическом тесте конденсаторов, возможно определение ёмкостей от \(1~pF\). Но для автоматического запуска
циклического теста ёмкость конденсатора должна быть больше \(25~pF\).
Cпециальные режимы могут быть завершены нажатием клавиши \textbf{ TEST}.
Тестер вернется к обычному режиму работы.
\item Из диалогового меню можно выбрать измерение частоты на порту PD4 для ATmega.
Разрешение составляет \(1~Hz\) для измеряемых частот выше \(33~kHz\).
Для более низких частот разрешение может быть до \(0,001~mHz\) с измерением среднего периода.
Вы должны ознакомиться с подразделом измерения частоты \ref{sec:frequency_counter} на 
странице \pageref{sec:frequency_counter} для уточнения деталей подключения сигнала частоты. 
\item Из меню, при отключенной функции последовательного порта, можно вызвать функцию измерения напряжения
до \(50~V\) при использовании делителя 10:1 на порту PC3. Если используется ATmega328 в корпусе PLCC, то для измерения
можнo использовать один из дополнительных портов вместе с UART. 
Если присутствует схема измерения стабилитронов (DC-DC преобразователь), измерение 
стабилитронов также возможно с помощью этой функции, нажав кнопку \textbf{ TEST}. 
\item Из меню можно выбрать функцию генератора частоты на тестовом контакте TP2 (PB2 порт ATmega).
В настоящее время можно предварительно выбрать частоты от \(10~Hz\) до \(2~MHz\).
\item Из диалогового меню функций можно выбрать вывод фиксированной частоты с возможностью 
выбора ширины импульса на тестовом контакте TP2 (PB2 порт ATmega). Ширина может быть 
увеличена на 1\% при кратковременном нажатии или на 10\% при более длительном нажатии кнопки \textbf{ TEST}.
\item Из диалогового меню функций можно запустить отдельное измерение ёмкости с измерением ESR.
Только конденсаторы от \(2~\mu F\) до \(50~mF\) могут быть измерены в схеме, так как используется низкое, около \(300~mV\)
напряжение.
\item Если Ваш контроллер ATmega имеет по крайней мере 32K флэш-памяти (Mega328), у Вас есть возможность использовать
метод дискретизации АЦП, который позволяет испытывать конденсаторы с ёмкостью меньше \(100~pF\) с разрешением \(0,01~pF\).
С применением этого же метода, можно также измерять индуктивность катушки меньше \(2~mH\) со значительно большей
точностью путем создания резонансного контура с параллельно включённым конденсатором известной ёмкости.

\end{enumerate}
Вы должны убедиться, что все конденсаторы разряжены перед началом любых измерений.

Тиристоры и симисторы могут быть обнаружены, если испытательный ток выше тока удержания. Некоторые тиристоры и 
симисторы нуждаются в более высоких токах, чем этот Тестер может обеспечить. Доступный ток тестирования только 
\(6~mA\)!
Также IGBT могут быть обнаружены, если \(5~V\) достаточно для их открытия.
Заметьте, что многие дополнительные функции могут быть доступны при использовании контроллеров с достаточным объемом памяти, таких как ATmega168.
Однако только при использовании контроллеров, в которых, по крайней мере \(32~kB\) флэш-памяти, таких как ATmega328 или ATmega1284 доступны все функции.

\vspace{1cm}
\textbf{{\Large Внимание:}} Перед подключением убедитесь, что  \textbf{ конденсаторы разряжены!} Тестер может быть 
повреждён и в выключенном состоянии. Есть только небольшая защита в портах ATmega.

Если требуется проверить элементы, установленные в схеме, то оборудование должно быть отсоединено от источника питания, 
и должна быть полная уверенность, что  \textbf{ остаточное напряжение} отсутствует в оборудовании. 
