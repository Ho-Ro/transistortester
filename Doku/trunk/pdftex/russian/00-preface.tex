Комментарий по этому вопросу:
\vspace*{0.3cm}
\\ При переводе на другой язык тексты на рисунках и диаграммах,
\\ которые на оригинальном английском, тоже переведены.
\\ - Добавлен подраздел 2.10 (клон из Hiland M644).
\\ - В раздел 4.1 был добавлен предметный программист.
\\ - И наконец, подраздел 4.2.1 (программирование под Linux) был добавлен,
\ 'Linux' новички 'также имеют успех.
\vspace*{0.3cm}
\\ Автор был проинформирован об этих мерах.
\\ К сожалению, насколько мне известно, документ еще не обновлялся.
\\ - Я не получил положительный ответ сам.
\\ - Поскольку я считаю, что дополнения важны для "новичков в Linux", это издание оправдано.
\vspace*{0.3cm}
\\ Оригинал, конечно, может быть достигнут ниже \cite{khk}.
\vspace*{0.3cm}
\\ 02/20/20
\\ bm-magic
\newpage
\section*{Вступление}
\subsection*{Основные мотивы}

Каждый радиолюбитель знает следующую задачу: Вы выпаяли транзистор из печатной платы или достали один из коробки. 
Если на нем есть маркировка, и у Вас уже есть паспорт или Вы можете получить документацию об этом элементе, то все 
в порядке. Но если документация отсутствует, то Вы понятия не имеете, что это за элемент. Традиционный подход измерения 
всех параметров сложный и трудоемкий. Элемент может быть N-P-N, P-N-P, N или P-канальным MOSFET транзистором и т.д. 
Идея Markus F. заключалась в том, чтобы переложить ручную работу на AVR микроконтроллер.
\subsection*{Начало моей работы над проектом}

Моя работа с программным обеспечением Тестера от Markus F.\cite{Frejek} началась, потому что у меня были проблемы с 
моим программатором. Я купил печатную плату и элементы, но не смог запрограммировать EEprom ATmega8 с драйвером Windows 
без сообщения об ошибке. Поэтому я взял программное обеспечение от Markus F. и изменил все обращения из памяти EEprom 
к Flash памяти. Анализируя программное обеспечение для того, чтобы сохранить память в других местах программы, у меня 
появилась идея изменить результат функции ReadADC из единиц АЦП на милливольты (\(mV\)). Размерность в \(mV\) необходима 
для любого вывода значения напряжения. Если функция ReadADC возвращает значения непосредственно в \(mV\), я могу 
сохранять преобразования для каждого выходного значения. Размерность в \(mV\) можно получить, если суммировать 
результаты 22 показаний АЦП, сумму умножить на 2 и разделить на 9. Таким методом максимальное значение получится 
\begin{math}\frac{1023\cdot22\cdot2}{9} = 5001\end{math},  что идеально соответствует нужной размерности измеренных 
значений напряжения в \(mV\). Кроме того дополнительно была надежда, что увеличение, от передискретизации, 
разрешения АЦП может способствовать улучшению считанного с АЦП напряжения, как описано в AVR121 \cite{AVR121}. 
В оригинальной версии функция ReadADC накапливается результат 20 измерений АЦП и делится потом на 20, так что результат 
равен оригинальному разрешению АЦП. Т.е., по этому пути повышение разрешения АЦП невозможно. Так что я должен был 
сделать небольшую работу, чтобы изменить функцию ReadADC, а это заставило проанализировать всю программу и изменить все 
\inquotes{if statements} в программе, где запрашиваются значения напряжения. Но это было только началом моей работы!\\

Появлялось все больше и больше идей, чтобы сделать измерения более быстрыми и точными. Кроме того хотелось расширить 
диапазон измерений сопротивлений и ёмкостей. Формат вывода информации на LCD-дисплей был изменен, теперь для диодов, 
резисторов и конденсаторов используются символы, а не текст. Для получения дополнительной информации необходимо 
ознакомиться со списком доступных функций в главе \ref{sec:features}. Планируемые работы и новые идеи представлены 
в главе \ref{sec:todo}. Кстати, теперь я могу программировать EEprom ATmega в операционной системе Linux без ошибок.\\

Здесь я хотел бы поблагодарить разработчика и автора программного обеспечения Markus Frejek, который предоставил 
возможность продолжить начатую им работу. Кроме того, я хотел бы сказать спасибо авторам многочисленных обсуждений 
на форуме, которые помогли мне найти новые задачи, слабые места и ошибки. Далее я хотел бы поблагодарить Markus 
Reschke, который разрешил мне публиковать его яркие версии программного обеспечения на сервере SVN. 
Кроме того, некоторые идеи и программные модули Markus R. были интегрированы в мою собственную версию программного 
обеспечения.

Также Wolfgang SCH. проделана большая работа по адаптации проекта под дисплей с контроллером ST7565. Большое спасибо ему
за адаптацию микропрограммы 1.10k к текущей версии.

Я должен поблагодарить также Asco B., который разработал новую печатную плату для повторения другими 
радиолюбителями. Следующую благодарность я хотел бы отправить Dirk W., который разработал порядок сборки этой печатной 
платы. У меня никогда не хватило бы времени заниматься всеми этими вещами одновременно с моими разработками 
программного обеспечения. Отсутствие времени не позволяет и в дальнейшем развивать программное обеспечение на том же 
уровне. Спасибо за многочисленные предложения по улучшению Тестера членам местного отделения 
\inquotes{Deutscher Amateur Radio Club (DARC)} из Lennestadt.
Кроме того, я хотел бы сказать спасибо за интеграцию метода дискретизации радиолюбителя \inquotes{Pieter-Tjerk (PA3FWM)}.
С помощью этого метода измерения маленьких значений емкости и индуктивности заметно улучшено.
На завершение, спасибо Nick L из Украины, за поддержку идей своими прототипами плат, предложение некоторых дополнений и
поддержку изменений в русской документации. 
